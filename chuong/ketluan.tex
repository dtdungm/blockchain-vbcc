%!TEX root = ../luanvan.tex
\chapter{Kết luận}
Qua quá trình nghiên cứu, cùng với sự giúp đỡ tận tình của giáo viên hướng dẫn, luận văn đã cơ bản hoàn thành được mục tiêu nghiên cứu, bao gồm một số kết quả sau đây:

\begin{enumerate}

\item Tìm hiểu nghiệp vụ quản lý và văn bản pháp lý về việc quản lý VBCC hiện hành theo quy định của pháp luật và tại Trung tâm Tin học Trường Đại học An Giang; Nghiên cứu tổng quan cơ sở lý thuyết mật mã, công nghệ blockchain và mô hình mạng Hyperledger Fabric.

\item Xây dựng website tương tác với người sử dụng trong việc cấp phát và xác thực chứng chỉ.

\end{enumerate}

\textbf{Hạn chế của đề tài}

Hạn chế của đề tài là chỉ dùng dịch vụ chứng thư số của Hyperledger Fabric và chứng thư số tự cấp trong hệ thống. Phạm vi nghiên cứu giới hạn gồm 3 bên tham gia: đơn vị cấp, bên xác minh và sinh viên. Tuy nhiên, cài đặt máy chủ hạ tầng khóa công khai và dịch vụ chứng thư số ở ngoài thực tế là công việc phức tạp và liên quan nhiều vấn đề bảo mật an toàn thông tin cần được quan tâm kỹ lưỡng.

Ngoài ra, dữ liệu nhập vào chuỗi khối đòi hỏi tính chính xác và tin cậy. Do đó đề tài cần tiếp tục nghiên cứu ứng dụng công nghệ blockchain trong quy trình tổ chức thi để có thông tin chính xác từ ban đầu đến khi cấp chứng chỉ. Thông tin cần được theo dõi khách quan, đảm bảo tin cậy cho người có VBCC, cơ quan quản lý và các tổ chức có liên quan.

Đề tài còn hạn chế là hệ thống Blochchain triển khai trên một máy, chưa đề xuất được mô hình mạng Blockchain phù hợp yêu cầu phân tán. 

\textbf{Định hướng nghiên cứu tiếp theo}

Ngoài những hạn chế trên, chắc chắn đề tài còn có nhiều thiếu sót. Do đó, đề tài sẽ tiếp tục việc nghiên cứu, cải tiến sau: (1) Nghiên cứu các thành phần của Hyperledger Fabric để ứng dụng nhiều tính năng hơn do nền tảng này cung cấp. (2) Nghiên cứu mở rộng các quy trình trong công tác tổ chức thi, liên quan đến cấp chứng chỉ. (3) Cải tiến giao diện người dùng giúp thuận tiện trong quản lý VBCC.
